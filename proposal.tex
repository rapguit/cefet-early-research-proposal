\documentclass[12pt]{article}

\usepackage{sbc-template}

\usepackage{graphicx,url}

\usepackage[brazil]{babel}   
%\usepackage[latin1]{inputenc}  
\usepackage[utf8]{inputenc}  
% UTF-8 encoding is recommended by ShareLaTex

     
\sloppy

\title{Padrões frequentes em comércio eletrônico}

\author{Raphael Correia de Souza Fialho\inst{1}}


\address{Programa de Pós-Graduação em Ciência da Computação -- CEFET/RJ\\
  CEP 20.271-110 -- Rio de Janeiro -- RJ -- Brasil
\email{raphael.fialho@b2wdigital.com}}


\begin{document} 

\maketitle

\begin{abstract}
   E-commerce websites use the A/B testing technique to decide how and what functionalities should be presented to end users, but this test does not identify the people profiles of who access the website, making it difficult to know the actual acceptance of the product for the desired audience.
   This research project proposes the development of an application that aims to categorize and distinguish the user profiles that access the website, allowing the business owner to be able to direct their e-commerce to the desired target audience without suffering distortions of the general appearance. The counselor pleaded is Prof. Dr. Eduardo Bezerra da Silva with coorientation of Prof. Dr. Eduardo Soares Ogasawara whose research project is Data and Application Management and the line of research is the Applications in Deep Learning and Pattern Recognition.
\end{abstract}
     
\begin{resumo} 
  Em websites de comércio eletrônico utiliza-se a técnica de teste A/B para decidir como e quais funcionalidades serão apresentadas aos usuários finais, porém este teste não identifica os tipos de pessoas que acessam o website, dificultando saber a real aceitação do produto para o público desejado.
  Este projeto de pesquisa propõe o desenvolvimento de uma aplicação que visa categorizar e distinguir os perfis de usuários que acessam o website, possibilitando que o dono do negócio consiga direcionar seu comércio eletrônico para o público-alvo desejado sem sofrer as distorções do aspecto geral. O orientador pleiteado é o Prof. Dr. Eduardo Bezerra da Silva com coorientação do Prof. Dr. Eduardo Soares Ogasawara cujo projeto de pesquisa é a Gerência de Dados e Aplicações e a linha de pesquisa é a Aplicações em Aprendizagem Profunda (Deep Learning) e Reconhecimento de Padrões.
\end{resumo}


\section{Introdução}

O Teste A/B é uma técnica de \textit{design} para decidir quais características causam maior aprovação dos usuários entre duas variantes: A e B. Duas versões de um mesmo \textit{website} são disponibilizados para seus usuários finais, onde para cada um é exibido apenas uma das versões, assim, é avaliado o grau de interesse e envolvimento de cada versão, medindo e avaliando a aceitação das funcionalidades.

A configuração experimental mais simples é avaliar um fator com dois níveis, um controle (versão A) e um tratamento (versão B). O controle é normalmente a versão padrão e o tratamento é a mudança que é testada. Essa configuração é comumente chamada de teste A/B. É comum ser estendido por vários níveis, muitas vezes referidos como testes de divisão A / B / N. Um experimento com
múltiplos fatores são referidos como \textit{Multivariable} (ou Multivariável) \cite{kohavi2010online}.

\begin{figure}[ht]
\centering
\includegraphics[width=.8\textwidth]{fig2.png}
\caption{Estrutura em alto nível de um experimento online. Usuários são divididos entre o sistema de controle e de tratamento. Interações dos usuários são instrumentadas, analisadas e comparadas. Análise no fim do experimento. Adaptado de \cite{kohavi2010online}.}
\label{fig:onlineExp1}
\end{figure}

O mapa \textit{heatmaps} destaca a influência do conteúdo da página onde as pessoas tendem a concentrar a visão, conforme a Figura \ref{fig:heatmap1}. Este é apenas alguns dos muitos métodos possíveis para entender as interações do usuário, e desenvolver hipóteses para testes controlados. Estudos de tração ocular mostram como as pessoas guiam sua atenção em uma página \cite{goward:13}.

\begin{figure}[ht]
\centering
\includegraphics[width=.5\textwidth]{fig1.jpg}
\caption{Exemplo de heatmap. Áreas avermelhadas caracterizam zona de maior atenção pelos usuários. Adaptado de \cite{goward:13}.}
\label{fig:heatmap1}
\end{figure}

Através das informações do Teste A/B, mapa \textit{heatmaps} e outras provenientes dos dados de acesso capturados das entradas das requisições na página, como \textit{logs} das máquinas servidoras, podemos extrair, avaliar e categorizar automaticamente o perfil do usuário que acessa o \textit{website}.

Após esta análise, será possível identificar quais os perfis de usuários que mais estão sendo influenciados positivamente ou negativamente, para a dada hipótese testada sobre as funcionalidades incluídas no \textit{website}.

\section{Objetivos}
\subsection{Objetivo Geral}
The first page must display the paper title, the name and address of the
authors, the abstract in English and ``resumo'' in Portuguese (``resumos'' are
required only for papers written in Portuguese). The title must be centered
over the whole page, in 16 point boldface font and with 12 points of space
before itself. Author names must be centered in 12 point font, bold, all of
them disposed in the same line, separated by commas and with 12 points of
space after the title. Addresses must be centered in 12 point font, also with
12 points of space after the authors' names. E-mail addresses should be
written using font Courier New, 10 point nominal size, with 6 points of space
before and 6 points of space after.

\subsection{Objetivo Específico}
The abstract and ``resumo'' (if is the case) must be in 12 point Times font,
indented 0.8cm on both sides. The word \textbf{Abstract} and \textbf{Resumo},
should be written in boldface and must precede the text.

\section{Método de Pesquisa}

In some conferences, the papers are published on CD-ROM while only the
abstract is published in the printed Proceedings. In this case, authors are
invited to prepare two final versions of the paper. One, complete, to be
published on the CD and the other, containing only the first page, with
abstract and ``resumo'' (for papers in Portuguese).

\section{topico pra fazer 1 ou mais}

Section titles must be in boldface, 13pt, flush left. There should be an extra
12 pt of space before each title. Section numbering is optional. The first
paragraph of each section should not be indented, while the first lines of
subsequent paragraphs should be indented by 1.27 cm.

\section{Resultados esperados}

Figure and table captions should be centered if less than one line
(Figure~\ref{fig:exampleFig1}), otherwise justified and indented by 0.8cm on
both margins, as shown in Figure~\ref{fig:exampleFig2}. The caption font must
be Helvetica, 10 point, boldface, with 6 points of space before and after each
caption.

In tables, try to avoid the use of colored or shaded backgrounds, and avoid
thick, doubled, or unnecessary framing lines. When reporting empirical data,
do not use more decimal digits than warranted by their precision and
reproducibility. Table caption must be placed before the table (see Table 1)
and the font used must also be Helvetica, 10 point, boldface, with 6 points of
space before and after each caption.

\begin{table}[ht]
\centering
\caption{Variables to be considered on the evaluation of interaction
  techniques}
\label{tab:exTable1}
\smallskip
\begin{tabular}{|l|c|c|}
\hline
& Value 1 & Value 2\\[0.5ex]
\hline
&&\\[-2ex]
Case 1 & 1.0 $\pm$ 0.1 & 1.75$\times$10$^{-5}$ $\pm$ 5$\times$10$^{-7}$\\[0.5ex]
\hline
&&\\[-2ex]
Case 2 & 0.003(1) & 100.0\\[0.5ex]
\hline
\end{tabular}
\end{table}

\section{Conclusão}

All images and illustrations should be in black-and-white, or gray tones,
excepting for the papers that will be electronically available (on CD-ROMs,
internet, etc.). The image resolution on paper should be about 600 dpi for
black-and-white images, and 150-300 dpi for grayscale images.  Do not include
images with excessive resolution, as they may take hours to print, without any
visible difference in the result. \cite{starner1997augmented}.

\section{References}

Bibliographic references must be unambiguous and uniform.  We recommend giving
the author names references in brackets, e.g. \cite{knuth:84},
\cite{boulic:91}, and \cite{smith:99}.

The references must be listed using 12 point font size, with 6 points of space
before each reference. The first line of each reference should not be
indented, while the subsequent should be indented by 0.5 cm.

\bibliographystyle{sbc}
\bibliography{sbc-template}

\end{document}
